\documentclass[11pt]{article}

\usepackage{times}
\usepackage{alltt}
\usepackage{hyperref}
\input{decls-common}

%% Page layout
\oddsidemargin 0pt
\evensidemargin 0pt
\textheight 600pt
\textwidth 469pt
\setlength{\parindent}{0em}
\setlength{\parskip}{1ex}

\begin{document}

\title{CSCI340, Spring 2015\\
Assignment 1: Writing a Kernel Module
}

\author{California State University - Chico\\
  By Bryan Dixon\\
}
\date{\emph{Due Date: Sunday, February 8, 2015 11:59pm}}


\maketitle

\section*{Introduction}
The purpose of this assignment is for you to become more familiar with how kernel modules are written and aspects of the Linux operating system using these modules. 

\section*{Logistics}

The only ``hand-in'' will be electronic.  Any clarifications and revisions to the assignment will
be posted on my web page and announced to the class via Piazza.

\section*{Hand Out Instructions}

For this assignment you will want to use a virtual machine (VM); however, if you are running a native install of the 64bit Ubuntu 14.04.1 LTS you should be able to do this without using a VM. I would recommend using a VM as we are going to be modifying privileged code and you could potentially corrupt your native system if you aren't using a VM. 

There are no handout files for this assignment; however, on my webpage for this assignment there is a provided set of files for the hello world kernel module that you should build first to familiarize yourself with the basics of compiling \& installing a compiled kernel module. 


\section*{Kernel Modules}

There are two main ways to add code to the Linux kernel. One way is to choose or add code to compile into the kernel during the compilation process. The other method is to add the code to the Linux kernel while it is running, which is what a loadable kernel module is \cite{tldp}.

For more information on Linux kernel modules, I highly recommend reading this extremely good introduction to Linux Loadable Kernel Modules and how they are commonly used:

{\url{http://www.tldp.org/HOWTO/Module-HOWTO/x73.html}}


\section*{Your Task}

For this assignment you will be doing the following:

\begin{enumerate}
\item{Get Ubuntu 14.04.1 Linux kernel source}
\item{Compile the latest Linux kernel source for Ubuntu 14.04.1}
\item{Compile a Hello World kernel module}
\item{Write a kernel module to create and modify a /proc file}
\end{enumerate}

In this document we will walk through the steps to do items 1-3 above. The code for the Hello World kernel module can be found on my website along with this writeup. Details on item 4 are found later in this document. 



\section*{Compile 14.04.1 kernel from source}
Your first step will be to download and install the 64bit 14.04 Ubuntu server \cite{ubuntuserver} onto your computer or VM. If you need help with this step please ask me to show you in lab or come to my office hours.

Once Ubuntu 14.04 is installed, you will need to set up the build environment inside of this installation by running the following commands in a terminal window:

\begin{verbatim}
$ sudo apt-get update
$ sudo apt-get install build-essential
$ sudo apt-get build-dep linux-image-$(uname -r)
\end{verbatim}

These commands update the package list to make sure we have the most up-to-date list of packages, install some general-purpose build tools, and finally install the build dependencies for the Linux kernel itself. In this way, we insure all the necessary tools to download, build, and install the new Linux kernel are available on the system.

The \texttt{sudo} in the previous commands indicate we are invoking the given commands as the root user. \texttt{sudo} only works if your user account has sudoer privileges; if not, you will receive a message indicating the user is not in the sudoers file. This is usually not an issue in standard installation, but if you encounter this message, it is simple to give the current user permission to run commands with \texttt{sudo} \cite{sudoers}.

Once the build environment is set up, you will need to download the source code for the Linux kernel you are currently running (we aren't trying to compile and install a newer kernel, just re-compile the current kernel). Downloading the source code for the Linux kernel a simple process, and very common for people who are running user-built (instead of package-maintained) Linux distros. Gentoo Linux is an example of such a distro if you are interested \cite{gentoo}.

To get the source code for the currently running Linux kernel on Ubuntu 14.04, we will use \texttt{apt-get}, which will obtain the source for a specific binary package it provides: 

\begin{verbatim}
$ mkdir kernel-assignment
$ cd kernel-assignment
$ apt-get source linux-image-$(uname -r)
\end{verbatim}

It is also common practice to obtain the Linux kernel source by checking out the current kernel source from the source tree on the official Linux git repository, but this assignment was tested with the apt-get approach so that's what I'm giving for the instructions. The references contain a link to an Ubuntu wiki article about building your own kernel that has the git commands if you want to know what they are \cite{ubuntukernel}.

The \texttt{apt-get} command will take a few minutes to download the Linux source code. When the download process finishes, you should have some additional files and a source code folder in the current working directory:

\begin{verbatim}
$ ls -l
total 121428
drwxr-xr-x 26 root root      4096 Aug 21 12:00 linux-3.13.0
-rw-r--r--  1 root root   7902814 Aug 13 15:58 linux_3.13.0-34.60.diff.gz
-rw-r--r--  1 root root     11781 Aug 13 15:58 linux_3.13.0-34.60.dsc
-rw-r--r--  1 root root 116419243 Feb  3  2014 linux_3.13.0.orig.tar.gz
\end{verbatim}

We are not going to modify the kernel's configuration, so we can now move to building the new kernel. Ubuntu does this a bit differently than other kernels I've built, which usually have a \texttt{make} directive to make the configuration, which can be the default or modified by you, and a second \texttt{make} directive to build the kernel. In this case, we will build the kernel using the following commands:

\begin{verbatim}
$ cd linux-3.13.0
$ fakeroot debian/rules clean
$ fakeroot debian/rules binary-headers binary-generic
\end{verbatim}

The first step above changes the working directory to be the root of the kernel source tree, which is the \texttt{linux-3.13.0} folder in this case. The build commands will take quite a while to run; on my fastest computer this build took an hour and half to build the \texttt{.deb} files. 

When the build process is complete (hopefully without any errors), there will be numerous \texttt{.deb} files in the parent directory of the kernel source tree (this parent directory will be the \texttt{kernel-assignement} directory we created when downloading the kernel source code). The \texttt{.deb} files contain the compiled kernel, which we can now install and run (after rebooting). If you get errors during the compilation, please post about them in the class discussion board and see me in office hours or during lab so we can track down and fix the issue.

Now let's install and test the new kernel. To do this we will use the following commands:

\begin{verbatim}
$ sudo dpkg -i linux-headers-*.deb
$ sudo dpkg -i linux-image-*.deb
$ sudo reboot
\end{verbatim}

After the computer reboots, we'll be using the new kernel you compiled. You can now brag about your \emph{1337} or \emph{leet} status as a CS major and the fact you have compiled the Linux kernel from source. 

\section*{Compile Hello World kernel module}

Now let's get the Hello World kernel module source and Makefile files from my web server and work on compiling a Linux kernel module. You will need to download the {\it helloworld.tar} file from my website:

{\url{http://bryancdixon.com/site_media/Fall2014/CSCI340/helloworld.tar}}

You could download the file from the link above to your local computer, but I would recommend downloading it directly to your Ubuntu VM so you can make use of it with having to worry about copying the files onto the VM. To do this you can use the {\it wget} command with that URL as the argument to the command and it'll download the {\it helloworld.tar} file to your current working directory.

Once you have the tar file you will want to extract it:

\begin{verbatim}
~$ tar xvf helloworld.tar
x helloworld/
x helloworld/hello.c
x helloworld/Makefile
\end{verbatim}

You should now have a \texttt{helloworld} folder in your current working directory. At this point you'll want to likely take a look at the \texttt{helloworld.c} source file and the Makefile to familiarize yourself with the workings of these two files. These two files are also a good starting point for the final part of this assignment. 

To build the Hello World kernel module you will need to change your working directory to be in the \texttt{helloworld} folder. Building kernel module is as simple as just typing \texttt{make}:

\begin{verbatim}
$ cd helloworld
$ make
\end{verbatim}

\newpage

This will build numerous files:

\begin{verbatim}
~/helloworld$ ls
hello.c  hello.ko  hello.mod.c  hello.mod.o  hello.o  
Makefile  modules.order  Module.symvers
\end{verbatim}

The only generated file that we care about is the \texttt{hello.ko} file, which is a kernel object file. We can now install the generated helloworld kernel module by using the \texttt{insmod} command:

\begin{verbatim}
~/helloworld# insmod hello.ko
~/helloworld# rmmod hello
~/helloworld# dmesg | tail
...
[ 6814.354580] Hello world!
[ 6819.571911] Cleaning up module.
\end{verbatim}

In the above example, I inserted the Hello World kernel module, immediately removed it with the \texttt {rmmod} command, and finally inspected the \texttt{dmesg} output to find the printk statements that printed "Hello World!" when the module was installed and the cleanup message when the module was removed. In the above output, this process completed successfully.

You may see a warning in the \texttt{dmesg} output:

\begin{verbatim}
~# dmesg | tail
...
[  258.556284] hello: module verification failed: signature and/or  
	required key missing - tainting kernel
[  258.558168] Hello world!
[  265.987671] Cleaning up module.
\end{verbatim}

This warning can be safely ignored.

\section*{Write your own kernel module}

Now for the hard part: using the skills you've gained in this assignment so far, resources provided later in the hints section, and some details from lab you'll now need to write your own Linux kernel module to provide us a system statistic in a /proc system file \cite{cse551}. 

When we insert your module for grading it should create a new entry in the /proc filesystem called:

\begin{verbatim}
/proc/num_pagefaults
\end{verbatim}

\newpage

We should then be able to cat or examine the contents of that file and it should provide us the number of page faults that the operating system has handled since it booted. As an example:

\begin{verbatim}
~# ls -al /proc/num_pagefaults
ls: /proc/num_pagefaults: No such file or directory
~# insmod ./pagefault.ko 
~# ls -al /proc/num_pagefaults
-r--r--r--  1 root root 37 August 22 21:38 /proc/num_pagefaults
~# cat /proc/num_pagefaults 
658103
~# cat /proc/num_pagefaults 
658295
~# cat /proc/num_pagefaults 
658485
~# rmmod page_fault_module
~# ls -al /proc/num_pagefaults
ls: /proc/num_pagefaults: No such file or directory
~#
\end{verbatim}

It's worth thinking of this problem in a few steps:

\begin{enumerate}
\item Read about how the /proc filesystem works
\item Figure out how you write information to a /proc file
\item Write a kernel module that successfully prints a fixed string when one cat's the /proc/num\_pagefaults file. 
\item Locate the kernel code that generates page faults statistics
\item Write a kernel module that prints that statistic every time someone cat's the /proc/num\_pagefaults file.
\end{enumerate}

Good luck!

\section*{Hints}

There are quite a few hints for this assignment:

\begin{enumerate}
\item When compiling the Linux kernel, make sure the virtual disk for the VM has plenty of space. The default disk size of 20 GB is probably sufficient.
\item Take a snapshot of the VM once the base system is installed and configured. This snapshot will be quite handy if something gets fouled up during the process of building the kernel.
\item Take a snapshot of the VM before installing the new kernel. This snapshot will be quite handy if something gets fouled up during the process of installing the kernel.
\item Be sure to successfully complete the kernel compilation portion of the assignment before attempting to compile the Hello World kernel module. In particular, the header files for the Linux kernel must be installed for kernel modules to build correctly.
\item You will need to find the symbol that has been explicitly exported by the kernel to be accessible to kernel modules; not all functions and variables in the kernel code are accessible to kernel modules. The kernel uses the \emph{EXPORT\_SYMBOL} macro to export a particular symbol, so you need to find the specific symbol that's been exported as such that provides the page fault statistic we want.
\item You may need to declare your kernel module is licensed under the GPL open source license, as some kernel symbols are only accessible if you have declared your kernel module as being licensed under the GPL license. To declare it you only need to add a single line at the end of your kernel module code\cite{gpl}:

\begin{verbatim}
MODULE_LICENSE("GPL");
\end{verbatim}
\item The Linux kernel already includes the page fault statistic in a /proc file, along with numerous other statistics. It is useful to see how this is already done and see if you can modify it to make a new /proc file that contains the current number of page faults only. The following command provides a good reference for the comparing the kernel statistics with those of your kernel module:

\begin{verbatim}
~$ cat /proc/vmstat | grep pgfault
pgfault 2301445
\end{verbatim}
\end{enumerate}

\section*{Evaluation}

You will be graded based on your success in completing various steps of this assignment. The scoring for this assignment is as follows:

\begin{itemize}
\item 20\% If you successfully compile a new kernel from source
\item 40\% You also successfully compile the Hello World kernel module
\item 70\% You also successfully write your own kernel module that creates the correct /proc file with a fixed value reported.
\item 100\% You successfully get everything else working and your kernel module correctly writes the number of page faults to the /proc file requested. 
\end{itemize}


\section*{Hand In Instructions}

We'll be handing this assignment in through GIT, details for how this will work will be added here soon. 

\begin{thebibliography}{3}

\bibitem{cse551} Steve Gribble
	\newblock \emph{CSE551 Spring 2007 - Programming Assignment \#2}
	\newblock \url{https://courses.cs.washington.edu/courses/cse551/07sp/programming/a2.html}.
	\newblock Online; accessed 22-August-2014 

\bibitem{sudoers} Allowing other users to run sudo
	\newblock \url{https://help.ubuntu.com/community/RootSudo#Allowing_other_users_to_run_sudo}.
	\newblock Online; accessed 19-January-2015.	
	
\bibitem{gentoo} Gentoo Linux
	\newblock \url{https://www.gentoo.org/}.
	\newblock Online; accessed 21-August-2014.	
	
\bibitem{gpl} The GNU General Public License
	\newblock \url{http://www.gnu.org/copyleft/gpl.html}
	\newblock Online; accessed 23-August-2014 


\bibitem{ubuntukernel} Ubuntu Wiki
  \newblock ``Build Your Own Kernel''.
  \newblock
  \url{https://wiki.ubuntu.com/Kernel/BuildYourOwnKernel}.
  \newblock Online; accessed 21-August-2014.
  
\bibitem{ubuntuserver} Ubuntu
  \newblock ``Download Ubuntu Server 14.04.1 LTS''.
  \newblock
  \url{ http://www.ubuntu.com/download/server}.
  \newblock Online; accessed 21-August-2014.
  
 
   
  
\bibitem{tldp} Linux Loadable Kernel Module HOWTO
  \newblock ``Introduction to Linux Loadable Kernel Modules''.
  \newblock
  \url{http://www.tldp.org/HOWTO/Module-HOWTO/x73.html}.
  \newblock Online; accessed 21-August-2014.

\end{thebibliography}
\end{document}
